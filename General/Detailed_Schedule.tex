\documentclass{eecslides}
\mode<presentation>
%\usecolortheme{BBSDark}

%%------------------
% Language and font
\usepackage[english]{babel}
\usepackage[utf8]{inputenc}

%%------------------
\usepackage{graphicx}
\usepackage{color}
\usepackage{tikz}
\usetikzlibrary{calc, shapes, backgrounds, arrows}

% --- include packages
\usepackage{subfigure}
\usepackage{multicol}
\usepackage{amsthm}
\usepackage{mathrsfs}
\usepackage{amsmath}
\usepackage{amssymb}
\usepackage{enumitem}

%%------------------
\DeclareRobustCommand\refmark[1]{\textsuperscript{\ref{#1}}}

%%------------------ Tune the template
\setbeamertemplate{blocks}[rounded][shadow=false]

\title[Outline]{Outline}
\vspace{0.4cm}
\vspace{0.6cm}

%%% Begin slideshow

\begin{document}

\begin{frame}
  \titlepage
\end{frame}

\section{Schedule}

\begin{frame}

\frametitle{Typical day}

\large

\begin{itemize}[leftmargin=2.5cm]
  \item[\bf 7h00 - 8h30] \textcolor{blue}{Breakfast}
  \item[\bf 8h30 - 10h00] Theory
  \item[\bf 10h00 - 10h15] \textcolor{green!50!black}{Break}
  \item[\bf 10h15 - 12h00] Practice
  \item[\bf 12h00 - 13h30] \textcolor{blue}{Lunch}
  \item[\bf 13h30 - 15h30] Practice
  \item[\bf 15h30 - 15h45] \textcolor{green!50!black}{Break}
  \item[\bf 15h45 - 17h30] Project
  \item[\bf 17h30 - 18h00] Presentation - Real ecological illustration
  \item[\bf 18h00 - 19h00] \textcolor{green!50!black}{Free time}
  \item[\bf 19h00 - 20h00] \textcolor{blue}{Supper}
  \item[\bf After 20h00] \textcolor{green!50!black}{Free time}
\end{itemize}

\end{frame}

\frame{
\frametitle{Day 1}


\vspace{-0.5cm}

\begin{itemize}[leftmargin=2.5cm]
  \item[\bf 8h30 - 9h00] Introduction
  \begin{itemize}
    \item Typical day
    \item Weekly schedule
    \begin{itemize}
      \item Course activities (programming with R)
      \item Leisure activities
    \end{itemize}
    \item Project (very brief introduction)
    \item Questions, comments, things that people wants
  \end{itemize}
  \item[\bf 9h00 - 9h30] Discussion about statistical philosophy (frequentist, Bayesian, priors, likelihood \dots) and expectations
  \item[\bf 9h30 - 10h00] Probability theory part 1 (Kevin Cazelles)
  \item[\bf 10h00 - 10h30] Practice
  \item[\bf 10h15 - 11h00] Probability theory part 2 (Kevin Cazelles)
  \item[\bf 11h00 - 12h00] Practice
  \begin{itemize}
    \item Small exercice(s) to be carried out by hand
  \end{itemize}
  \item[\bf 13h30 - 15h30] Probability theory part 3 (Kevin Cazelles)
  \begin{itemize}
    \item Slightly more complex exercises that requires the computer
  \end{itemize}
  \item[\bf 15h45 - 17h30] Project
  \begin{itemize}
    \item Discuss project in more details (students should know what they will do for the project by 17h30)
  \end{itemize}
  \item[\bf 17h30 - 18h00] Presentation
  \begin{itemize}
    \item Presentation of Kevin work on co-occurrence (15 to 30 slides ecological context with more technical details)
  \end{itemize}
\end{itemize}
}


% Very simple question/problem (linear model, ANOVA)
% Forest data ecological

\frame{
\frametitle{Day 2}

\large

\vspace{-0.5cm}

\begin{itemize}[leftmargin=2.5cm]
  \item[\bf 8h30 - 10h00] Maximum Likelihood and Optimization (Dominique Gravel)
  {\normalsize
  \begin{itemize}
    \item Why the likelihood is important in Bayesian statistics?
    \item Simulated annealing
    \item Other optimization techniques? (Nelder-Mead simplex?)
  \end{itemize}
  }
  \item[\bf 10h15 - 12h00] Practice
  {\normalsize
  \begin{itemize}
    \item Code your own simulated annealing algorithm ``by hand''
  \end{itemize}
  }
  \item[\bf 13h30 - 15h30] Practice
  {\normalsize
  \begin{itemize}
    \item Apply the code to fit a model
  \end{itemize}
  }
  \item[\bf 15h45 - 17h30] Project
  {\normalsize
  \begin{itemize}
    \item Work what has been learned so far in the project
  \end{itemize}
  }
  \item[\bf 17h30 - 18h00] Presentation
  {\normalsize
  \begin{itemize}
    \item Presentation of a research paper on the topic of the day  (15 to 30 slides ecological context with more technical details)
  \end{itemize}
  }
\end{itemize}
}

\frame{
\frametitle{Day 3}

\vspace{-0.5cm}

\begin{itemize}[leftmargin=2.5cm]
  \item[\bf 8h30 - 10h00] \textcolor{blue}{MCMC} and \textcolor{orange}{Model Evaluation} (Guillaume Blanchet)
  {\normalsize
  \begin{itemize}
    \item \textcolor{blue}{Link with Likelihood (Frequentist) and Bayesian}
    \item \textcolor{blue}{Why MCMC ? Link with the likelihood}
    \item \textcolor{blue}{Metropolis-Hasting}
      \begin{itemize}
        \item \textcolor{blue}{Single component adaptive Metropolis (SCAM)}
      \end{itemize}
    \item \textcolor{blue}{Gibb sampling}
    \item \textcolor{blue}{Conjugate prior distribution} 
    \item \textcolor{blue}{Other things, HMC, INLA (very stiff learning curve... Ben Bolker told me this),...} 
    \item \textcolor{orange}{trace and density plots}
    \item \textcolor{orange}{various convergence diagnostics}
  \end{itemize}
  }
  \item[\bf 10h15 - 12h00] Practice
  {\normalsize
  \begin{itemize}
    \item Code an MCMC for a simple example (e.g. curves with three bumps as a likelihood)
  \end{itemize}
  }
  \item[\bf 13h30 - 15h30] Practice
  {\normalsize
  \begin{itemize}
    \item Continue... Think of a more complex example (2D...)
  \end{itemize}
  }
  \item[\bf 15h45 - 17h30] Project
  {\normalsize
  \begin{itemize}
    \item Work what has been learned so far in the project
  \end{itemize}
  }
  \item[\bf 17h30 - 18h00] Presentation
  {\normalsize
  \begin{itemize}
    \item Presentation of Nature Ecology and Evolution work (15 to 30 slides ecological context with more technical details)
  \end{itemize}
  }
\end{itemize}
}

\frame{
\frametitle{Day 4}

\small
\vspace{-0.5cm}

\begin{itemize}[leftmargin=2.5cm]
  \item[\bf 8h30 - 10h00] Hierarchical modelling (Guillaume Blanchet \& Matthew Talluto?)
  \begin{itemize}
    \item What is a hierarchical model?
    \begin{itemize}
      \item Link with random effect and mixed models
    \end{itemize}
    \item Start from a generalized linear model without any hierarchy and build from therm
    \item Ecological illustration to work through
    \item Directed acyclic graph (DAG)
    \begin{itemize}
      \item Data
      \item Parameters to estimate
      \item Priors
    \end{itemize}
  \end{itemize}
  \item[\bf 10h15 - 12h00] Practice
  \begin{itemize}
    \item Answer ecological question with data using what was learned so far
  \end{itemize}
  \item[\bf 13h30 - 15h30] Practice
  \begin{itemize}
    \item Continue... Think of a more complex example
  \end{itemize}
  \item[\bf 15h45 - 17h30] Project
  \begin{itemize}
    \item Work what has been learned so far in the project
  \end{itemize}
  \item[\bf 17h30 - 18h00] Presentation
  \begin{itemize}
    \item Presentation of JSDM work (15 to 30 slides ecological context with more technical details)
  \end{itemize}
\end{itemize}
}

\frame{
\frametitle{Day 5}

\small
\vspace{-0.5cm}

\begin{itemize}[leftmargin=2.5cm]
  \item[\bf 8h30 - 10h00] Model Comparison (Matthew Talluto \& Guillaume Blanchet?)
  \begin{itemize}
    \item AIC (see recent papers by Taper and Ponciano [2016]), BIC, DIC, \dots
    \item Model selection ?
    \item \dots
  \end{itemize}
  \item[\bf 10h15 - 12h00] Practice
  \begin{itemize}
    \item Answer ecological question with data using what was learned so far
  \end{itemize}
  \item[\bf 13h30 - 15h30] Practice
  \begin{itemize}
    \item Continue... Think of a more complex example
  \end{itemize}
  \item[\bf 15h45 - 17h30] Project
  \begin{itemize}
    \item Work what has been learned so far in the project
  \end{itemize}
  \item[\bf 17h30 - 18h00] Presentation
  \begin{itemize}
    \item Presentation of a research paper on the topic of the day (to be determined)
  \end{itemize}
\end{itemize}
}

\end{document}

- Problem usefulness of using Bayesian
- Dataset
- Question
- 

Send small 2500 words
- Paper in short format (Nature type manuscript)
- Methods as appendix
