\documentclass[12]{article}
\usepackage[utf8]{inputenc}
\usepackage{authblk}
\usepackage{geometry}
\usepackage{amsmath}
\usepackage{hyperref}
\usepackage[french]{babel}
\usepackage{url}

\geometry{letterpaper,margin=2.5cm}

\hypersetup
{
    colorlinks = true, linkcolor = blue, citecolor = blue, urlcolor = blue,
}

%\def\labelitemi{$\bullet$}

\title{ECL 707 (Maîtrise) - ECL 807 (Doctorat) \\ Cours d'été en statistiques Bayésiennes pour écologistes}
\date {\today}
\author[1]{Dominique Gravel}
\author[1]{Guillaume Blanchet}
\author[2]{Matthew Talluto}
\affil[1]{Départment de biologie, Université de Sherbrooke}
\affil[2]{Laboratoire d'écologie Alpine, Université Grenoble 1}

\begin{document}

	\maketitle

	%-----------------------------
	\section*{Objectif général}

  Les statistiques Bayésiennes sont de plus en plus utilisées en écologie et ce à
  travers l'ensemble des sous-domaines de la discipline; autant pour étudier le
  mouvement d'individus d'une espèces, les facteurs qui structurent une
  population ou une communauté, les comportements d'animaux, les réseaux
  trophiques... Au terme de ce cours, l'étudiant sera en mesure de comprendre
  les bases théoriques des statistiques Bayésiennes et sera en mesure de faires
  des analyses simples (e.g. régression linéaire) et plus complexes (e.g.
  modèles hierarchiques) à travers le paradigme Bayésien. En plus d'avoir acquis
  des connaissances théoriques en statistiques Bayésiennes, l'étudiant sera en
  mesure d'implémenter diverses méthodes d'analyses présentées pendant le cours.

	%-----------------------------
	\section*{Objectifs spécifiques}

	Au terme de ce cours, l'étudiant sera en mesure de:

	\begin{itemize}
	\renewcommand{\labelitemi}{$\bullet$}

  \item Utiliser la théorie des probabilité pour construire un modèle statistique ;

  \item Écrire la fonction de vraisemblance d'un modèle et trouver son maximum via diverses techniques d'optimisation ;

  \item Comprendre les bases fondamentales des statistiques Bayésiennes ;

	\item Estimer les paramètres d'un modèle Bayésien en utilisant diverses approches;

	\item Définir et étudier un modèle hierarchique ;

	\item Comparer des modèles à travers le paradigme Bayésien ;

	\end{itemize}

	%-----------------------------
	\section*{Pré-requis}

	La réalisation de ce cours requiert les pré-requis suivants :

  \begin{itemize}
    \renewcommand{\labelitemi}{$\bullet$}
	   \item Connaissance intermédiaire de la programmation scientifique ;

     \item Connaissance avancée des modèles linéaires généralisés et mixtes ;
  \end{itemize}

    %-----------------------------

  \section*{Approche pédagogique}

  Les séances seront constituées de courtes leçons magistrales sur des notions
  de bases, entre-coupées d'exercices spécifiques destinés à pratiquer les
  éléments enseignés. Les séances seront complémentés de discussions sur les
  caractéristiques des approches présentées et de leur utilité en écologie. Les
  étudiants seront invités à réaliser un projet intégrateur d'analyse de données
  sur l'ensemble de la semaine de formation intensive.

	L'ensemble du matériel du cours sera disponible sur un dépôt git à l'adresse :\\
	\url{https://github.com/EcoNumUdS/ECL707}

	%-----------------------------
	\section*{Contenu}

	\subsection*{Jour 1 : Théorie des probabilités}

%	\begin{itemize}
%	\renewcommand{\labelitemi}{$\bullet$}
%		\item
%	\end{itemize}

  \subsection*{Jour 2 : Maximum de vraisemblance et optimisation}

%	\begin{itemize}
%	\renewcommand{\labelitemi}{$\bullet$}
%		\item
%	\end{itemize}

  \subsection*{Jour 3 : Méthode de Monte-Carlo par chaînes de Markov et évaluation de modèles}

%	\begin{itemize}
%	\renewcommand{\labelitemi}{$\bullet$}
%		\item
%	\end{itemize}

  \subsection*{Jour 4 : Modèles hierarchiques}

%	\begin{itemize}
%	\renewcommand{\labelitemi}{$\bullet$}
%		\item
%	\end{itemize}

  \subsection*{Jour 5 : Comparaison de modèles}

%  \begin{itemize}
%	\renewcommand{\labelitemi}{$\bullet$}
%		\item
%	\end{itemize}

	%-----------------------------
	\section*{Évaluation}

	L'évaluation portera sur la réalisation d'un projet intégrateur d'analyse de
	données. Ils devront remettre un rapport accompagné du script utilisé pour
	réaliser les analyses. Les étudiants auront l'opportunité d'utiliser leurs
	propres données s'ils le souhaitent ou pourront définir un projet sur des
	données en accès libre.

%%% Comment ça fonctionne pour une cours d'été ? Est-ce qu'il y en a une ?

\end{document}
