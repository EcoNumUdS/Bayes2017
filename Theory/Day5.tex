\documentclass{eecslides}
\mode<presentation>
%\usecolortheme{BBSDark}

%%------------------
% Language and font
\usepackage[english]{babel}
\usepackage[utf8]{inputenc}

%%------------------
\usepackage{graphicx}
\usepackage{color}
\usepackage{tikz}
\usetikzlibrary{calc, shapes, backgrounds, arrows}

% --- include packages
\usepackage{subfigure}
\usepackage{multicol}
\usepackage{amsthm}
\usepackage{mathrsfs}
\usepackage{amsmath}
\usepackage{amssymb}
\usepackage{enumitem}

%%------------------
\DeclareRobustCommand\refmark[1]{\textsuperscript{\ref{#1}}}

%%------------------ Tune the template
\setbeamertemplate{blocks}[rounded][shadow=false]

\title[Day 5]{Model Comparison}
\vspace{0.4cm}
\vspace{0.6cm}

%%% Begin slideshow

\begin{document}

\frame{
  \titlepage
}

\section{Introduction}

\frame{
	\frametitle{Day 5 Schedule}
	
	\small
	\vspace{-0.5cm}
	
	\begin{itemize}[leftmargin=2.5cm]
		\item[\bf 8h30 - 10h00] Model Comparison (And other topics) (Matthew Talluto)
		\item[\bf 10h15 - 12h00] Simple example
		\begin{itemize}
			\item Perhaps the easiest thing will be to compute DIC and a naive AIC (using all pseudoparameters) for the hierarchical model from day 4
		\end{itemize}
		\item[\bf 13h30 - 17h30] Student presentations
		\item[\bf 17h30 - 18h00] Project work and wrap up
	\end{itemize}
}

\frame{
\frametitle{Introduction to Model Comparison}
			\begin{itemize}
	\item why compare models?
	\item importance of identifying goals in model comparison/evaluation
	\item predictive performance? hypothesis testing? reduction of overfitting (related to prediction)? simplification (if n is limiting?)
	\item first piece of advice if you want to do model comparison: DON'T
	\begin{itemize}
		\item rather compare effect sizes, standard deviations
		\item prefer methods of model predictive performance if prediction is the goal
	\end{itemize}
\end{itemize}
}
%
\frame{
	\frametitle{Model comparison methods}
	\begin{itemize}
		\item posterior predictive density (good for models with same size, same n)
		\item Bayes factor (similar to likelihood ratio) (perhaps this is where MAP estimation should be mentioned, as it has bearing on BF - BF uses the MAP)
			\begin{itemize}
				\item can compute the MAP from posterior samples (just look for the mode); however this is computationally intensive
				\item another approach is to use another optimization method (e.g., LA, SA) to estimate the MAP, compute BF, then take posterior samples for best model
		\end{itemize}
	\end{itemize}
}

\frame{
\frametitle{Information criteria}
	\begin{itemize}
		\item DIC, WAIC (BIC?)
		\item why/why not to use
		\item why not AIC? (what is k for a hierarchical model?)
		\item DIC is not a significance test, and IC thresholds aren't meaningful
		\item Taper and Ponciano [2016] (if time)
	\end{itemize}

}
\frame{
	\frametitle{other topics (to be mentioned, but not discussed in detail)}
	\begin{itemize}
		\item Bayesian model averaging
		\item RJ-MCMC
		\item Other algorithms? HMC?
\end{itemize}
}



%
%
%\section{Example}
%
%\frame{
%\frametitle{Ecological example discussed during the other day with an additional layer of complexity}
%For example add another species of tree or a patch level effect
%}
%
%\section{Model building}
%
%\frame{
%\frametitle{Conceptually using Direct Acyclic Diagram (DAG)}
%}
%
%\frame{
%\frametitle{The importance of defining the pieces properly}
%\begin{itemize}
%  \item Priors
%  \item Parameters
%  \item Data
%\end{itemize}
%}
%
%\frame{
%\frametitle{Deciding on the technical aspects of the model}
%\begin{itemize}
%  \item Distribution from which to sample
%  \item Properties of priors
%\end{itemize}
%}
%
%\frame{
%\frametitle{Defining the model mathematically}
%}
%
%\frame{
%\frametitle{How to estimate the model?}
%Based on what we learned so far
%}
%
%\frame{
%\frametitle{Writing down the sampler}
%Maybe this is a bit too difficult ??
%}
%
%\section{Parameter estimation}
%
%\frame{
%\frametitle{Different way to estimate the parameters}
%}
%
%\section{Parameter estimation}
%
%\frame{
%\frametitle{MCMC}
%}
%
%\frame{
%\frametitle{Check for convergence}
%}
%

\end{document}
