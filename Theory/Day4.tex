\documentclass{eecslides}\usepackage[]{graphicx}\usepackage[]{color}
%% maxwidth is the original width if it is less than linewidth
%% otherwise use linewidth (to make sure the graphics do not exceed the margin)
\makeatletter
\def\maxwidth{ %
  \ifdim\Gin@nat@width>\linewidth
    \linewidth
  \else
    \Gin@nat@width
  \fi
}
\makeatother

\definecolor{fgcolor}{rgb}{0.345, 0.345, 0.345}
\newcommand{\hlnum}[1]{\textcolor[rgb]{0.686,0.059,0.569}{#1}}%
\newcommand{\hlstr}[1]{\textcolor[rgb]{0.192,0.494,0.8}{#1}}%
\newcommand{\hlcom}[1]{\textcolor[rgb]{0.678,0.584,0.686}{\textit{#1}}}%
\newcommand{\hlopt}[1]{\textcolor[rgb]{0,0,0}{#1}}%
\newcommand{\hlstd}[1]{\textcolor[rgb]{0.345,0.345,0.345}{#1}}%
\newcommand{\hlkwa}[1]{\textcolor[rgb]{0.161,0.373,0.58}{\textbf{#1}}}%
\newcommand{\hlkwb}[1]{\textcolor[rgb]{0.69,0.353,0.396}{#1}}%
\newcommand{\hlkwc}[1]{\textcolor[rgb]{0.333,0.667,0.333}{#1}}%
\newcommand{\hlkwd}[1]{\textcolor[rgb]{0.737,0.353,0.396}{\textbf{#1}}}%
\let\hlipl\hlkwb

\usepackage{framed}
\makeatletter
\newenvironment{kframe}{%
 \def\at@end@of@kframe{}%
 \ifinner\ifhmode%
  \def\at@end@of@kframe{\end{minipage}}%
  \begin{minipage}{\columnwidth}%
 \fi\fi%
 \def\FrameCommand##1{\hskip\@totalleftmargin \hskip-\fboxsep
 \colorbox{shadecolor}{##1}\hskip-\fboxsep
     % There is no \\@totalrightmargin, so:
     \hskip-\linewidth \hskip-\@totalleftmargin \hskip\columnwidth}%
 \MakeFramed {\advance\hsize-\width
   \@totalleftmargin\z@ \linewidth\hsize
   \@setminipage}}%
 {\par\unskip\endMakeFramed%
 \at@end@of@kframe}
\makeatother

\definecolor{shadecolor}{rgb}{.97, .97, .97}
\definecolor{messagecolor}{rgb}{0, 0, 0}
\definecolor{warningcolor}{rgb}{1, 0, 1}
\definecolor{errorcolor}{rgb}{1, 0, 0}
\newenvironment{knitrout}{}{} % an empty environment to be redefined in TeX

\usepackage{alltt}
\mode<presentation>
%\usecolortheme{BBSDark}

%%------------------
% Language and font
\usepackage[english]{babel}
\usepackage[utf8]{inputenc}

%%------------------
\usepackage{graphicx}
\usepackage{color}
\usepackage{tikz}
\usetikzlibrary{calc, shapes, backgrounds, arrows}

% --- include packages
\usepackage{subfigure}
\usepackage{multicol}
\usepackage{amsthm}
\usepackage{mathrsfs}
\usepackage{amsmath}
\usepackage{amssymb}
\usepackage{enumitem}

%%------------------
\DeclareRobustCommand\refmark[1]{\textsuperscript{\ref{#1}}}

%%------------------ Tune the template
\setbeamertemplate{blocks}[rounded][shadow=false]

\title[Day 4]{Hierarchical models}
\vspace{0.4cm}
\vspace{0.6cm}

%%% Begin slideshow
\IfFileExists{upquote.sty}{\usepackage{upquote}}{}
\begin{document}

\frame{
  \titlepage
}

\section{Introduction}

\frame{
\frametitle{What is a hierarchical model?}
\framesubtitle{How important is elevation in defining sugar maple distribution on mont Sutton?}


\begin{columns}[T]
\begin{column}{0.6\textwidth}

  {\large\bf Global Model}
  
  \vspace{-0.2cm}

  {\large $$P(y=1) = \beta x + \epsilon$$}
  {\large\bf Seperate model}
  
  \vspace{-0.2cm}
  
  {\large $$\textcolor{orange}{P(y=1) = \beta_1 x + \epsilon_1}$$}
  {\large $$\textcolor{green!75!blue}{P(y=1) = \beta_2 x + \epsilon_2}$$}
  {\large $$\textcolor{red!8!green!57!Blue}{P(y=1) = \beta_3 x + \epsilon_3}$$}

  \vspace{-0.8cm}

 where
 \begin{itemize}
  \item[$y$] is the distribution of sugar maple
  \item[$x$] is elevation
  \textcolor{red}{\large\item[$\beta$] is the importance of elevation}
  \item[$\epsilon$] is the model residuals
 \end{itemize}
\end{column}
\begin{column}{0.5\textwidth}
\vspace{-0.5cm}
\begin{knitrout}
\definecolor{shadecolor}{rgb}{0.969, 0.969, 0.969}\color{fgcolor}
\includegraphics[width=\maxwidth]{figure/acerSacc-1} 

\end{knitrout}
\end{column}
\end{columns}
}

\frame{
\frametitle{Why are hierarchical model worth studying?}
}


\section{Example}

\frame{
\frametitle{Ecological example discussed during the other day with an additional layer of complexity}
For example add another species of tree or a patch level effect
}

\section{Model building}

\frame{
\frametitle{Conceptually using Direct Acyclic Diagram (DAG)}
}

\frame{
\frametitle{The importance of defining the pieces properly}
\begin{itemize}
  \item Priors
  \item Parameters
  \item Data
\end{itemize}
}

\frame{
\frametitle{Deciding on the technical aspects of the model}
\begin{itemize}
  \item Distribution from which to sample
  \item Properties of priors
\end{itemize}
}

\frame{
\frametitle{Defining the model mathematically}
}

\frame{
\frametitle{How to estimate the model?}
Based on what we learned so far
}

\frame{
\frametitle{Writing down the sampler}
Maybe this is a bit too difficult ??
}

\section{Parameter estimation}

\frame{
\frametitle{Different way to estimate the parameters}
}

\section{Parameter estimation}

\frame{
\frametitle{MCMC}
}

\frame{
\frametitle{Check for convergence}
}


\end{document}
