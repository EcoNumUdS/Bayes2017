\documentclass{eecslides}
\mode<presentation>
%\usecolortheme{BBSDark}

%%------------------
% Language and font
\usepackage[english]{babel}
\usepackage[utf8]{inputenc}

%%------------------
\usepackage{graphicx}
\usepackage{color}
\usepackage{tikz}
\usetikzlibrary{calc, shapes, backgrounds, arrows}

% --- include packages
\usepackage{subfigure}
\usepackage{multicol}
\usepackage{amsthm}
\usepackage{mathrsfs}
\usepackage{amsmath}
\usepackage{amssymb}
\usepackage{enumitem}

%%------------------
\DeclareRobustCommand\refmark[1]{\textsuperscript{\ref{#1}}}

%%------------------ Tune the template
\setbeamertemplate{blocks}[rounded][shadow=false]

\title[Day 4]{Hierarchical models}
\vspace{0.4cm}
\vspace{0.6cm}

%%% Begin slideshow

\begin{document}

\frame{
  \titlepage
}

\section{Introduction}

\frame{
\frametitle{What is a hierarchical model?}
\begin{itemize}
\end{itemize}
}

\frame{
\frametitle{Why are hierarchical model worth studying?}
}


\section{Example}

\frame{
\frametitle{Ecological example discussed during the other day with an additional layer of complexity}
For example add another species of tree or a patch level effect
}

\section{Model building}

\frame{
\frametitle{Conceptually using Direct Acyclic Diagram (DAG)}
}

\frame{
\frametitle{The importance of defining the pieces properly}
\begin{itemize}
  \item Priors
  \item Parameters
  \item Data
\end{itemize}
}

\frame{
\frametitle{Deciding on the technical aspects of the model}
\begin{itemize}
  \item Distribution from which to sample
  \item Properties of priors
\end{itemize}
}

\frame{
\frametitle{Defining the model mathematically}
}

\frame{
\frametitle{How to estimate the model?}
Based on what we learned so far
}

\frame{
\frametitle{Writing down the sampler}
Maybe this is a bit too difficult ??
}

\section{Parameter estimation}

\frame{
\frametitle{Different way to estimate the parameters}
}

\section{Parameter estimation}

\frame{
\frametitle{MCMC}
}

\frame{
\frametitle{Check for convergence}
}


\end{document}
