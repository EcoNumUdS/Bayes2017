\documentclass{eecslides}
\mode<presentation>
%\usecolortheme{BBSDark}

%%------------------
% Language and font
\usepackage[english]{babel}
\usepackage[utf8]{inputenc}

%%------------------
\usepackage{graphicx}
\usepackage{color}
\usepackage{tikz}
\usetikzlibrary{calc, shapes, backgrounds, arrows}

% --- include packages
\usepackage{subfigure}
\usepackage{multicol}
\usepackage{amsthm}
\usepackage{mathrsfs}
\usepackage{amsmath}
\usepackage{amssymb}
\usepackage{enumitem}

%%------------------
\DeclareRobustCommand\refmark[1]{\textsuperscript{\ref{#1}}}

%%------------------ Tune the template
\setbeamertemplate{blocks}[rounded][shadow=false]

\title[Intro]{Introduction}
\vspace{0.4cm}
\vspace{0.6cm}

%%% Begin slideshow

\begin{document}

\frame{
  \titlepage
}

\section{Introduction}


\frame{
\frametitle{Language}

Même si l'école d'été est chapeauté par l'Université de Sherbrooke, la partie magistrale de l'école (y compris les notes de cours) et les discussions avec l'ensemble du groupe seront en anglais.

\vspace{0.5cm}

Par contre, les intéractions un à un peuvent se faire au tant en anglais qu'en français; c'est comme vous le souhaitez.

------------------

Eventhough the summer school is officially under Université de Sherbrooke, the magistral part of the school (including class notes) and discussion with all the group will be in English.

\vspace{0.5cm}

However, one-on-one interactions can be either in French or in English, whatever is easier for you.

\vspace{0.5cm}

{\large\bf The main reason for this choice is that some of the students do not speak French}
}

\frame{
\frametitle{Round table}

\large {\bf Who are you and who is your neighbour?}

\begin{itemize}
  \item Who are you?
  \item Who are we?
  \item Where are you from?
  \item What are your interests?
\end{itemize}

\vspace{0.5cm}

\pause
\large {\bf Why are you here?}

\begin{itemize}
  \item Forced ? ;o)
  \item Why Bayesian statistics?
  \item What do you know about Bayesian statistics?
  \item What did you think of the papers we asked you to read?
\end{itemize}

}

\frame{
\frametitle{Objective of the course}

\large

We are aware that the background, expertise and interest of everyone is different. So, our general goal is that everyone learns and make some progress in their understanding of Bayesian statistics.

\vspace{0.2cm}

{\bf What we want expect from you}

\vspace{0.2cm}

\begin{itemize}
  \item[\textbullet] Use proability theory to construct statistical models
  \item[\textbullet] Write the likelihood of a model and find its maximum through different optimization techniques
  \item[\textbullet] Understand the fondamental aspects of Bayesian statistics
  \item[\textbullet] Estimate the parameter of a Bayesian model using different approaches
  \item[\textbullet] Understand the rudiment of hierarchical model
  \item[\textbullet] Compare models through the Bayesian paradigm
\end{itemize}
}

\frame{
\frametitle{Course project}

\begin{description}
  \item[Introduction] Briefly give some background to the project and explain the problem that will be approached (at most 1 page)
  \item[Data] Present the data used to approach the problem
  \item[Method] Describe in {\bf details} the methods to use and explain how it can answer the question define in the Introduction
  \item[Result] Perform the analysis. The {\bf code} used and the output should be handed with the course project
  \item[Discussion] Breifly discuss the results (at most 1 page)
\end{description}

\textcolor{red}{The project should be writen in the form of a Nature paper}

\begin{itemize}
  \item It needs to be sent to us by {\bf\textcolor{red}{Friday August 17\superscript{th} 2018}}
  \item It can be done in French or English
  \item It can be a solo or team (of at most {\bf 3 students}) project
\end{itemize}
}

\frame{
\frametitle{Course presentation}

\large
\begin{itemize}
  \item It will be a short presentation of no more than {\bf\textcolor{red}{10 minutes + Questions}}
  \item It will present the general outline of your project
  \item You will present it on {\bf\textcolor{red}{Friday}}
\end{itemize}

{\bf Structure of the presentation}

\vspace{0.2cm}

\begin{itemize}
  \item Context
  \item Problem approached
  \item Explain why you want to use Bayesian statistics
  \item Describe the data
  \item Present the likelihood function
  \item Present the priors
\end{itemize}

}

\frame{
\frametitle{Typical day}
\begin{itemize}[leftmargin=2.5cm]
  \item[\bf 7h00 - 8h30] \textcolor{blue}{Breakfast}
  \item[\bf 8h30 - 10h00] Class
  \item[\bf 10h00 - 10h15] \textcolor{green!50!black}{Break}
  \item[\bf 10h15 - 12h00] Class
  \item[\bf 12h00 - 13h30] \textcolor{blue}{Lunch}
  \item[\bf 13h30 - 15h30] Class
  \item[\bf 15h30 - 15h45] \textcolor{green!50!black}{Break}
  \item[\bf 15h45 - 17h30] Class
  \item[\bf 17h30 - 18h00] Discussion - Real ecological problem
  \item[\bf 18h00 - 19h00] \textcolor{green!50!black}{Free time}
  \item[\bf 19h00 - 20h00] \textcolor{blue}{Supper}
  \item[\bf After 20h00] \textcolor{green!50!black}{Free time}
\end{itemize}

This is not meant to be followed strickly and will likely vary depending on the topic of the day

}

\frame{
\frametitle{General approach to teaching}

Each day will be divided in roughly the same way

\begin{itemize}
  \item Theory and exercices
  \item Application
  \item Problem to solve
  \item Project
\end{itemize}

}

\frame{
\frametitle{Schedule of the week}

\begin{description}
  \item[Day 1]
  \begin{description}
    \item[Theory] Probability Theory
    \item[Discussion] Defining the prior for the problem of ecological interactions
    \item[Application] Understanding co-distribution
  \end{description}
  \item[Day 2]
  \begin{description}
    \item[Theory] Likelihood estimation
    \item[Discussion] Code a simulated annealing function to model tree distribution
    \item[Application] Fitting a probabilistic method to presence-only data
  \end{description}
  \item[Day 3]
  \begin{description}
    \item[Theory] Monte Carlo Markov Chains (MCMC)
    \item[Discussion] Code different types of MCMC algorithm (Metropolis-Hastings and Gibbs sampler) to model the distribution of the sugar maple on mont Sutton and test model quality
    \item[Application] Modelling range dynamics
  \end{description}
\end{description}
}
\frame{
\frametitle{Schedule of the week}
\begin{description}
  \item[Day 4]

  \begin{description}
    \item[Theory] Hierarchical models
    \item[Discussion] Code Gibbs samplers to construct a univariate mixed model and a multivariate model to model the distribution of the sugar maple and the american beech on mont Sutton
    \item[Application] Hierarchical modelling og species community
  \end{description}
  \item[Day 5]
  \begin{description}
    \item[Theory] miscellaneous interesting things about Bayesian modelling (model comparison, different ways to estimate models,\dots)
    \item[Discussion]
    \item[Presentation]
  \end{description}
\end{description}
}

\frame{
\frametitle{Evaluation}

\begin{description}
  \item[Presentation] 25\%
  \item[Project report] 75\%
\end{description}
}

\frame{
\frametitle{Logistics}

\Huge
Steve !
Steve !
Steve !
Steve !
}

\end{document}
